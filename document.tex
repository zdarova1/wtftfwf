\documentclass[12pt]{article}
%\usepackage{fontspec}
\usepackage{amsmath}
\usepackage{amsfonts}
\usepackage{graphicx}
\usepackage{enumitem}
\usepackage{amssymb}
\usepackage{relsize}
\usepackage{ upgreek }
\usepackage[utf8]{inputenc}
\usepackage[russian]{babel}
\usepackage[left=2cm,right=2cm,
top=2cm,bottom=2cm,bindingoffset=0cm]{geometry}
\pagestyle{empty}
\setlength{\parskip}{10pt}
%\setmainfont{Times New Roman}



\begin{document}
	
\begin{center}
	МИНИСТЕРСТВО ОБРАЗОВАНИЯ И НАУКИ РОССИЙСКОЙ ФЕДЕРАЦИИ \\ ГОСУДАРСТВЕННОЕ БЮДЖЕТНОЕ ОБРАЗОВАТЕЛЬНОЕ УЧРЕЖДЕНИЕ \\ 
	ВЫСШЕГО ПРОФЕССИОНАЛЬНОГО ОБРАЗОВАНИЯ
	\vskip 1.5cm
	«Московский государственный технический \\
	университет имени Н.Э. Баумана» \\
	(МГТУ им. Н.Э. Баумана)
	\vskip 1.5cm
	ФАКУЛЬТЕТ ФУНДАМЕНТАЛЬНЫЕ НАУКИ \\
	КАФЕДРА \\
	«ВЫЧИСЛИТЕЛЬНАЯ МАТЕМАТИКА И МАТЕМАТИЧЕСКАЯ ФИЗИКА»
	\vskip 0.4cm
	Направление: \textbf{Математика и компьютерные науки}
	\vskip 0.4cm
	Дисциплина: Численные методы
	\vskip 0.4cm
	Домашняя работа №1-3 \\
	1. Методы простой итерации и Зейделя \\
	2. Методы касательных и секущих, метод деления отрезка пополам \\
	Группа ФН11-51Б
	\vskip 0.2cm
	Вариант 1
	
	
	\vskip 1.5cm
	\begin{flushright}
		Студент: Авилов О.Д.
		
		\vskip 1.5cm
		
		Преподаватель: Кутыркин В.А
	\end{flushright}
	Оценка:
	
	\begin{figure}[b]
		\begin{center}
			Москва 2022
		\end{center}
	\end{figure}
	
\end{center}

% Конец титульника
\newpage

\begin{center}
	\textbf{\textit{Задание 1.1}}
\end{center}
Используя метод простой итерации с нулевым начальным вектором, найти приближённое
решение СЛАУ:
$	\mathbf{A \cdot {}^{\mathsmaller{>}} x = {}^\mathsmaller{>}b}$
, с матрицей, имеющей диагональное преобладание.
Абсолютная погрешность приближённого решения не должна превышать величины 0,01.
Предполагается, что все компоненты решения заданной СЛАУ равны единице. Матрица
A
этой СЛАУ приведена ниже в зависимости от варианта задания.
Кроме того, используя неравенство (3), найти в методе простой итерации число шагов,
необходимое для того чтобы гарантировать абсолютную погрешность приближённого
решения не более 0,01. Сравнить это расчётное количество шагов с реальным количеством
шагов, обеспечившим заданную погрешность.

$N = 1, \beta = 1-0,02(49-51) = 1,04$

\begin{equation*}
	A = \left(
	\begin{array}{cccc}
		10\beta & 1 & 2 & 3\\
		1 & 10\beta & 3 & 2\\
		2 & 3 & 10\beta & 1\\
		3 & 2 & 1 & 10\beta\\
	\end{array}
	\right)
	=
	\left(
	\begin{array}{cccc}
		10,4 & 1 & 2 & 3\\
		1 & 10,4 & 3 & 2\\
		2 & 3 & 10,4 & 1\\
		3 & 2 & 1 & 10,4\\
	\end{array}
	\right)
\end{equation*}
\begin{center}
	\textbf{\textit{Решение}}
\end{center}
\begin{equation*}
\text{Найдем вектор } b = A \cdot x = \left(
\begin{array}{cccc}
	10,4 & 1 & 2 & 3\\
	1 & 10,4 & 3 & 2\\
	2 & 3 & 10,4 & 1\\
	3 & 2 & 1 & 10,4\\
\end{array}
\right) \cdot
\left(
\begin{array}{c}
	1\\ 1\\ 1\\	1\\
\end{array}\right) = 
\left(
\begin{array}{c}
	16,4\\ 16,4\\ 16,4\\ 16,4\\
\end{array}\right)
\end{equation*}\\
Приведем СЛАУ к виду ${}^{\mathsmaller{>}} x = 
F \cdot {}^{\mathsmaller{>}} x + {}^{\mathsmaller{>}} g$, где
$F = E - D \cdot A, ||F|| < 1 $ и $ {}^{\mathsmaller{>}} g = D \cdot {}^{\mathsmaller{>}} b$\\
Т.к. матрица A имеет диагональное преобладание, то можно D выбрать в виде:
\begin{equation*}
	D = \left(
	\begin{array}{cccc}
		\dfrac 1 {10,4} & 0 & 0 & 0\\
		0 & \dfrac 1 {10,4} & 0 & 0\\
		0 & 0 & \dfrac 1 {10,4} & 0\\
		0 & 0 & 0 & \dfrac 1 {10,4}\\
	\end{array}
	\right)
	=
	\left(
	\begin{array}{cccc}
		0,09615 & 0 & 0 & 0\\
		0 & 0,09615 & 0 & 0\\
		0 & 0 & 0,09615 & 0\\
		0 & 0 & 0 & 0,09615\\
	\end{array}
	\right)
\end{equation*}
\begin{equation*}
	\text{Тогда } F = E - D \cdot A = \left(
	\begin{array}{cccc}
		0 & -0,0962 & -0,1923 & -0,2885 \\
		-0,0962 & 0 & -0,2885 & -0,1923 \\
		-0,1923 & -0,2885 & 0 & -0,0962 \\
		-0,2885 & -0,1923 & -0,0962 & 0 \\
	\end{array}
	\right), \hspace{0.2cm} {}^{\mathsmaller{>}} g = 
	\left(
	\begin{array}{c}
		1,57692 \\ 1,57692 \\ 1,57692 \\ 1,57692 \\ 
	\end{array}\right)
\end{equation*}
\begin{equation*}
	\text{Положим начальный вектор итераций } {}^{\mathsmaller{>}} x_0 = 
	 \left(
	 \begin{array}{c}
	 	0 \\ 0 \\ 0\\ 0\\
	 \end{array}
	 \right)
\end{equation*}
Рассчитаем количество шагов, гарантирующее абсолютную погрешность приближённого решения не более 0,01, используя неравенство:
\begin{equation*}
||{}^{\mathsmaller{>}} x _{(k)} - {}^{\mathsmaller{>}} x_*|| \leqslant
\dfrac{||F||^k}{1- ||F||} \cdot || {}^{\mathsmaller{>}} g || + 
||F||^k \cdot ||{}^{\mathsmaller{>}} x_{(0)}||
\end{equation*}
Решая данное неравенство получаем количество итераций $ k \geqslant 11 $ \\
Тогда равносильная СЛАУ:
\begin{equation*}
	{}^{\mathsmaller{>}} x = F \cdot {}^{\mathsmaller{>}} x + {}^{\mathsmaller{>}} g = \left(
	\begin{array}{cccc}
		0 & -0,0962 & -0,1923 & -0,2885 \\
		-0,0962 & 0 & -0,2885 & -0,1923 \\
		-0,1923 & -0,2885 & 0 & -0,0962 \\
		-0,2885 & -0,1923 & -0,0962 & 0 \\
	\end{array}
	\right) \cdot {}^{\mathsmaller{>}} x + \left(
	\begin{array}{c}
		1,57692 \\ 1,57692 \\ 1,57692 \\ 1,57692 \\ 
	\end{array}\right)
\end{equation*}
Рабочая формула метода простой итерации: $  {}^{\mathsmaller{>}} x_{(k+1)}= F \cdot 
 {}^{\mathsmaller{>}} x_{(k)} +  {}^{\mathsmaller{>}} g $
Итерируем по шагам:
\begin{equation*}
	{}^{\mathsmaller{>}} x_{1}= F \cdot 
	{}^{\mathsmaller{>}} x_{0} +  {}^{\mathsmaller{>}} g = 
	\left(
	\begin{array}{c}
		1,57692 \\ 1,57692 \\ 1,57692 \\ 1,57692 \\
	\end{array}
	\right), \hspace{0.2cm}
	\varepsilon_1 = 0,57692 > 0,01
\end{equation*}
\begin{equation*}
	{}^{\mathsmaller{>}} x_{2}= F \cdot 
	{}^{\mathsmaller{>}} x_{1} +  {}^{\mathsmaller{>}} g = 
	\left(
	\begin{array}{c}
		0,66716 \\ 0,66716 \\ 0,66716 \\ 0,66716 \\ 
	\end{array}
	\right), \hspace{0.2cm}
	\varepsilon_2 = 0,33284 > 0,01
\end{equation*}
\begin{equation*}
	{}^{\mathsmaller{>}} x_{3}= F \cdot 
	{}^{\mathsmaller{>}} x_{2} +  {}^{\mathsmaller{>}} g = 
	\left(
	\begin{array}{c}
		1,19202 \\ 1,19202 \\ 1,19202 \\ 1,19202 \\ 
	\end{array}
	\right), \hspace{0.2cm}
	\varepsilon_3 = 0,19202 > 0,01
\end{equation*}
\begin{equation*}
	{}^{\mathsmaller{>}} x_{4}= F \cdot 
	{}^{\mathsmaller{>}} x_{3} +  {}^{\mathsmaller{>}} g = 
	\left(
	\begin{array}{c}
		0,88922 \\ 0,88922 \\ 0,88922 \\ 0,88922 \\ 
	\end{array}
	\right), \hspace{0.2cm}
	\varepsilon_4 = 0,11078 > 0,01
\end{equation*}
\begin{equation*}
	{}^{\mathsmaller{>}} x_{5}= F \cdot 
	{}^{\mathsmaller{>}} x_{4} +  {}^{\mathsmaller{>}} g = 
	\left(
	\begin{array}{c}
		1,06391 \\ 1,06391 \\ 1,06391 \\ 1,06391 \\ 
	\end{array}
	\right), \hspace{0.2cm}
	\varepsilon_5 = 0,06391 > 0,01
\end{equation*}
\begin{equation*}
	{}^{\mathsmaller{>}} x_{6}= F \cdot 
	{}^{\mathsmaller{>}} x_{5} +  {}^{\mathsmaller{>}} g = 
	\left(
	\begin{array}{c}
		0,96313 \\ 0,96313 \\ 0,96313 \\ 0,96313 \\ 
	\end{array}
	\right), \hspace{0.2cm}
	\varepsilon_6 = 0,03687 > 0,01
\end{equation*}
\begin{equation*}
	{}^{\mathsmaller{>}} x_{7}= F \cdot 
	{}^{\mathsmaller{>}} x_{6} +  {}^{\mathsmaller{>}} g = 
	\left(
	\begin{array}{c}
		1,02127 \\ 1,02127 \\ 1,02127 \\ 1,02127 \\ 
	\end{array}
	\right), \hspace{0.2cm}
	\varepsilon_7 = 0,02127 > 0,01
\end{equation*}
\begin{equation*}
	{}^{\mathsmaller{>}} x_{8}= F \cdot 
	{}^{\mathsmaller{>}} x_{7} +  {}^{\mathsmaller{>}} g = 
	\left(
	\begin{array}{c}
		0,98773 \\ 0,98773 \\ 0,98773 \\ 0,98773 \\ 
	\end{array}
	\right), \hspace{0.2cm}
	\varepsilon_8 = 0,01227 > 0,01
\end{equation*}
\begin{equation*}
	{}^{\mathsmaller{>}} x_{9}= F \cdot 
	{}^{\mathsmaller{>}} x_{8} +  {}^{\mathsmaller{>}} g = 
	\left(
	\begin{array}{c}
		1,00708 \\ 1,00708 \\ 1,00708 \\ 1,00708 \\ 
	\end{array}
	\right), \hspace{0.2cm}
	\varepsilon_9 = 0,00708 < 0,01
\end{equation*}
Получили приближённое решение СЛАУ $
	{}^{\mathsmaller{>}} x_{9}= 
	\left(
	\begin{array}{c}
		1,00708 \\ 1,00708 \\ 1,00708 \\ 1,00708 \\ 
	\end{array}
	\right)$
c погрешностью не более 0,01 за 9 итераций (k = 9)
\begin{center}
	\textbf{\textit{Результаты}}
\end{center}
Методом простых итераций было получено решение СЛАУ на 9 итерации с точностью 0,01.
По формуле оценки точность решения не более 0,01 достигается на 11 итерации.
 
\begin{center}
	\textbf{\textit{Задание 1.2}}
\end{center}

Используя метод Зейделя с нулевым начальным вектором, найти приближённое решение
СЛАУ: $	\mathbf{A \cdot {}^{\mathsmaller{>}} x = {}^\mathsmaller{>}b}$
, с матрицей, имеющей диагональное преобладание. Абсолютная
погрешность приближённого решения не должна превышать величины 0,01.
Предполагается, что все компоненты решения заданной СЛАУ равны единице. Матрица
A
этой СЛАУ приведена в Таблицах 1а,б. Сравнить в методах простой итерации и
Зейделя количество шагов для достижения абсолютной погрешности, не превышающей
величины 0,01.

$N = 1, \beta = 1-0,02(49-51) = 1,04$

\begin{equation*}
	A = \left(
	\begin{array}{cccc}
		10\beta & 1 & 2 & 3\\
		1 & 10\beta & 3 & 2\\
		2 & 3 & 10\beta & 1\\
		3 & 2 & 1 & 10\beta\\
	\end{array}
	\right)
	=
	\left(
	\begin{array}{cccc}
		10,4 & 1 & 2 & 3\\
		1 & 10,4 & 3 & 2\\
		2 & 3 & 10,4 & 1\\
		3 & 2 & 1 & 10,4\\
	\end{array}
	\right)
\end{equation*}
\begin{center}
	\textbf{\textit{Решение}}
\end{center}
Аналогично заданию 1.1 найдем равносильную СЛАУ (такая же как в задании 1.1):
\begin{equation*}
	{}^{\mathsmaller{>}} x = F \cdot {}^{\mathsmaller{>}} x + {}^{\mathsmaller{>}} g = \left(
	\begin{array}{cccc}
		0 & -0,0962 & -0,1923 & -0,2885 \\
		-0,0962 & 0 & -0,2885 & -0,1923 \\
		-0,1923 & -0,2885 & 0 & -0,0962 \\
		-0,2885 & -0,1923 & -0,0962 & 0 \\
	\end{array}
	\right) \cdot {}^{\mathsmaller{>}} x + \left(
	\begin{array}{c}
		1,57692 \\ 1,57692 \\ 1,57692 \\ 1,57692 \\ 
	\end{array}\right)
\end{equation*}
Рабочая формула метода Зейделя: $ {}^{\mathsmaller{>}} y_{(k)} = P\cdot 
{}^{\mathsmaller{>}} y_{(k-1)} + Q \cdot {}^{\mathsmaller{>}} y_{(k)} +
{}^{\mathsmaller{>}} g, k \in \mathbb{N} $,
где
\begin{equation*}
	 B = \left(
\begin{array}{cccc}
	f^1_1 & 0 & \dots & 0 \\
	f^2_1 & f^2_2 & \ddots & \vdots \\
	\vdots & \ddots & \ddots & 0 \\
	f^n_1 & f^n_2 & \dots & f^n_n \\
\end{array}
\right), \hspace{0.2cm}
D = \left(
\begin{array}{cccc}
	f^1_1 & 0 & \dots & 0 \\
	0 & f^2_2 & \ddots & \vdots \\
	\vdots & \ddots & \ddots & 0 \\
	0 & \dots & 0 & f^n_n \\
\end{array}
\right), \hspace{0.2cm}
Q = B - D, P = F-Q, F = (f^i_j)^n_n
\end{equation*}
Найдем P и Q:
\begin{equation*}
	Q = \left(
	\begin{array}{cccc}
		0 & 0 & 0 & 0 \\
		-0,0962 & 0 & 0 & 0 \\
		-0,1923 & -0,2885 & 0 & 0 \\
		-0,2885 & -0,1923 & -0,0962 & 0 \\
	\end{array}
	\right), \hspace{0.2cm}
	P = \left(
	\begin{array}{cccc}
		0 & -0,0962 & -0,1923 & -0,2885 \\
		0 & 0 & -0,2885 & -0,1923 \\
		0 & 0 & 0 & -0,0962 \\
		0 & 0 & 0 & 0 \\
	\end{array}
	\right)
\end{equation*}
$ {}^{\mathsmaller{>}} y_{(k)} = (E-Q)^{-1} \cdot P\cdot 
{}^{\mathsmaller{>}} y_{(k-1)} + (E-Q)^{-1} \cdot
{}^{\mathsmaller{>}} g, k \in \mathbb{N} $, 
итерируем по шагам:
\begin{equation*}
	{}^{\mathsmaller{>}} y_{1}= 
	\left(
	\begin{array}{c}
		1,57692 \\ 1,4253 \\ 0,86253 \\ 0,76501 \\ 
	\end{array}
	\right), \hspace{0.2cm}
	\varepsilon_1 = 0,57692 > 0,01
\end{equation*}
\begin{equation*}
	{}^{\mathsmaller{>}} y_{2}= 
	\left(
	\begin{array}{c}
		1,05333 \\ 1,07972 \\ 0,98934 \\ 0,97031 \\ 
	\end{array}
	\right), \hspace{0.2cm}
	\varepsilon_2 = 0,07972 > 0,01
\end{equation*}
\begin{equation*}
	{}^{\mathsmaller{>}} y_{3}= 
	\left(
	\begin{array}{c}
		1,00295 \\ 1,0085 \\ 0,99984 \\ 0,99753 \\ 
	\end{array}
	\right), \hspace{0.2cm}
	\varepsilon_3 = 0,0085 < 0,01
\end{equation*}
Таким образом, необходимая точность достигается уже на 3 итерации.
\begin{center}
	\textbf{\textit{Результаты}}
\end{center}
Методом Зейделя было получено решение данной СЛАУ на 3 итерации с точностью 0,01. Следовательно, приближенное решение сходится к истинному быстрее, чем методом простых итераций.

\begin{center}
	\textbf{\textit{Задание 2}}
\end{center}

C погрешностью, не превосходящей величину $\varepsilon = 0,0001$
, найти все корни уравнения: 

$[N+5,2+(-1)^N\alpha] \cdot x^3 - [2N^2 + 10,4N + (-1)^{N+1}\alpha] \cdot x^2 -
N^2(N+5,2)(x-2N) +(-1)^N \alpha = 0$

Нарисовать график функции, стоящей в левой части уравнения. Используя этот
график отделить корни уравнения. Для определения левого корня использовать метод
касательных, правого – метод секущих. Для определения срединного корня использовать
метод деления отрезка пополам.

$N = 1, \alpha = 0,003 \cdot (51-50) = 0,003$

Наше уравнение:
\begin{equation*}
	6,197 \cdot x^3 - 12,403 \cdot x^2 -
	6,2 \cdot x + 12,397 = 0
\end{equation*}
\begin{center}
	\textbf{\textit{Решение}}
\end{center}
Построим график и найдем корни уравнения:
\begin{figure}[h]%current location
	\centering
	\scalebox{0.9}{\includegraphics{graph.png}}
\end{figure}
\begin{equation*}
	\begin{split}
		x_1 & = -0,9999193429 \\
		x_2 & = 0,9992753320 \\
		x_3 & = 2,002096327
	\end{split}
\end{equation*}
\begin{enumerate}[label=\textbf{\arabic*})] 
	\item \textbf{Определим левый корень методом касательных} \\
	Рабочая формула для метода касательных имеет вид:
	\begin{equation*}
		x_k = x_{k-1} - (f^\prime(x_{k-1}))^{-1} \cdot f(x_{k-1})
	\end{equation*}
	Возьмем отрезок $[-1,5; -0.5]$, на этом отрезке $f^\prime > 0$ и $f^{\prime \prime} < 0$, $f(-1,5) < 0, f(-0.5) > 0$, соответственно на этом отрезке есть 1 корень уравнения, функция монотонно возрастает, сохраняя при этом направление выпуклости. Возьмем в качестве начального приближения $x_0 = -1,5$ \\
	Получаем приближенные решения:
\begin{equation*}
	x_1 = -1,1276071154, \hspace{0.2cm}
	\varepsilon_1 = 0,1276877725 > 0,0001
\end{equation*}
\begin{equation*}
	x_2 = -1,0116152683, \hspace{0.2cm}
	\varepsilon_2 = 0,0116959254 > 0,0001
\end{equation*}
\begin{equation*}
	x_3 = -1,0000316716, \hspace{0.2cm}
	\varepsilon_3 = 0,0001123287 > 0,0001
\end{equation*}
\begin{equation*}
	x_4 = -0,9999193534, \hspace{0.2cm}
	\varepsilon_4 = 1,05e-08 < 0,0001
\end{equation*}
Получили необходимую точность на 4 итерации.
	\item \textbf{Определим правый корень методом секущих} \\
Рассмотрим отрезок $[1,5; 2,5]$. На этом отрезке выполняются соотношения: \\
$f(1,5) \cdot f(2,5) < 0 $ и $f(2,5) \cdot f^{\prime \prime} > 0 $ 

	Тогда рабочая формула для метода секущих имеет вид:
\begin{equation*}
	x_k = x_{k-1} - \dfrac{(b-k_{k-1})f(x_{k-1})}{f(b)-f(x_{k-1})}, (k \in N),
\end{equation*}
где $b = 2,5$. Возьмем в качестве начального приближения $x_0 = 1,5$ \\
Получаем приближенные решения:
\begin{equation*}
	x_1 = 1,6937628257, \hspace{0.2cm}
	\varepsilon_1 = 0,3083335013 > 0,0001
\end{equation*}
\begin{equation*}
	x_2 = 1,8394531563, \hspace{0.2cm}
	\varepsilon_2 = 0,1626431707 > 0,0001
\end{equation*}
\begin{equation*}
	x_3 = 1,9247925463, \hspace{0.2cm}
	\varepsilon_3 = 0,0773037807 > 0,0001
\end{equation*}
\begin{equation*}
	x_4 = 1,9674072239, \hspace{0.2cm}
	\varepsilon_4 = 0,0346891031 > 0,0001
\end{equation*}
\begin{equation*}
	x_5 = 1,9869570229, \hspace{0.2cm}
	\varepsilon_5 = 0,0151393041 > 0,0001
\end{equation*}
\begin{equation*}
	x_6 = 1,9955716259, \hspace{0.2cm}
	\varepsilon_6 = 0,0065247011 > 0,0001
\end{equation*}
\begin{equation*}
	x_7 = 1,9992997515, \hspace{0.2cm}
	\varepsilon_7 = 0,0027965755 > 0,0001
\end{equation*}
\begin{equation*}
	x_8 = 2,0009005181, \hspace{0.2cm}
	\varepsilon_8 = 0,0011958089 > 0,0001
\end{equation*}
\begin{equation*}
	x_9 = 2,001585522, \hspace{0.2cm}
	\varepsilon_9 = 0,000510805 > 0,0001
\end{equation*}
\begin{equation*}
	x_{10} = 2,0018782249, \hspace{0.2cm}
	\varepsilon_{10} = 0,0002181021 > 0,0001
\end{equation*}
\begin{equation*}
	x_{11} = 2,0020032195, \hspace{0.2cm}
	\varepsilon_{11} = 0,0000931075 < 0,0001
\end{equation*}
Получили необходимую точность на 11 итерации.
	\item \textbf{Определим срединный корень методом деления отрезка пополам} \\
Смотря на график, выберем отрезок $[0,5; 1,5]$, он нам подходит, т.к. $f(0,5) \cdot f(1,5) < 0 $. \\
Получаем приближенные решения:
\begin{equation*}
	x_1 = 1,0, \hspace{0.2cm}
	\varepsilon_1 = 0,000724668 > 0,0001
\end{equation*}
\begin{equation*}
	x_2 = 0,75, \hspace{0.2cm}
	\varepsilon_2 = 0,249275332 > 0,0001
\end{equation*}
\begin{equation*}
	x_3 = 0,875, \hspace{0.2cm}
	\varepsilon_3 = 0,124275332 > 0,0001
\end{equation*}
\begin{equation*}
	x_4 = 0,9375, \hspace{0.2cm}
	\varepsilon_4 = 0,061775332 > 0,0001
\end{equation*}
\begin{equation*}
	x_5 = 0,96875, \hspace{0.2cm}
	\varepsilon_5 = 0,030525332 > 0,0001
\end{equation*}
\begin{equation*}
	x_6 = 0,984375, \hspace{0.2cm}
	\varepsilon_6 = 0,014900332 > 0,0001
\end{equation*}
\begin{equation*}
	x_7 = 0,9921875, \hspace{0.2cm}
	\varepsilon_7 = 0,007087832 > 0,0001
\end{equation*}
\begin{equation*}
	x_8 = 0,99609375, \hspace{0.2cm}
	\varepsilon_8 = 0,003181582 > 0,0001
\end{equation*}
\begin{equation*}
	x_9 = 0,998046875, \hspace{0.2cm}
	\varepsilon_9 = 0,001228457 > 0,0001
\end{equation*}
\begin{equation*}
	x_{10} = 0,9990234375, \hspace{0.2cm}
	\varepsilon_{10} = 0,0002518945 > 0,0001
\end{equation*}
\begin{equation*}
	x_{11} = 0,9995117188, \hspace{0.2cm}
	\varepsilon_{11} = 0,0002363868 > 0,0001
\end{equation*}
\begin{equation*}
	x_{12} = 0,9992675781, \hspace{0.2cm}
	\varepsilon_{12} = 0,0000077539 < 0,0001
\end{equation*}
Получили необходимую точность на 12 итерации.
\end{enumerate}
\begin{center}
	\textbf{\textit{Результаты}}
\end{center}
Наибольшей скоростью сходимости из трех рассмотренных в данном задании методов для выбранных начальных значений и отрезков обладает метод касательных. Следующий по скорости - метод секущих. Наименьшей скоростью сходимости в данной задаче обладает метод деления отрезка пополам. 
\end{document}